%%%%%%%%%%%%%%%%%%%%%%%%%%%%%%%%%%%%%%%%%
% Simple Sectioned Essay Template
% LaTeX Template
%
% This template has been downloaded from:
% http://www.latextemplates.com
%
% Note:
% The \lipsum[#] commands throughout this template generate dummy text
% to fill the template out. These commands should all be removed when 
% writing essay content.
%
%%%%%%%%%%%%%%%%%%%%%%%%%%%%%%%%%%%%%%%%%

%----------------------------------------------------------------------------------------
%	PACKAGES AND OTHER DOCUMENT CONFIGURATIONS
%----------------------------------------------------------------------------------------

\documentclass[12pt]{article} % Default font size is 12pt, it can be changed here

\usepackage{geometry} % Required to change the page size to A4
\geometry{a4paper, left={40mm}, right={40mm}, top={25mm}, bottom={25mm},includehead,includefoot} % Set the page size to be A4 as opposed to the default US Letter

\usepackage{graphicx} % Required for including pictures

\usepackage{float} % Allows putting an [H] in \begin{figure} to specify the exact location of the figure
\usepackage{wrapfig} % Allows in-line images such as the example fish picture

\usepackage{lipsum} % Used for inserting dummy 'Lorem ipsum' text into the template

\usepackage{amsfonts}
\usepackage{amsmath}

\usepackage{listings}
\lstset{
	basicstyle=\footnotesize\ttfamily,%
	breaklines=true,%
	captionpos=b,                    % sets the caption-position to bottom
	tabsize=2,	                   % sets default tabsize to 2 spaces
	%keywordstyle=\color{red},
}

\usepackage{hyphenat}
\usepackage{hyperref}

\usepackage{xcolor}
\hypersetup{
    colorlinks,
    linkcolor=,
    %linkcolor={red!50!black},
    citecolor={blue!50!black},
    urlcolor={blue!80!black}
}

\usepackage[sorting=none]{biblatex}
\addbibresource{bibliografia.bib}

\usepackage{pgfplots}

% ---------------------------------------------------------------------------------------
%	TIPOGRAFÍA
% ---------------------------------------------------------------------------------------

\usepackage[no-math]{fontspec}
\setmainfont[
	Path=./../fuentes/, 
	UprightFont=* Regular, 
	ItalicFont=* Italic,
	BoldFont=* Bold,
  	BoldItalicFont=* Bold Italic,
]{Equity Text A}
\setsansfont[
	Path=./../fuentes/, 
	UprightFont=* Regular, 
	ItalicFont=* Italic,
	BoldFont=* Bold,
  	BoldItalicFont=* Bold Italic,
]{Concourse T3}
\setmonofont[
	Path=./../fuentes/, 
	UprightFont=* Regular,% 
	%ItalicFont=* Italic,
	BoldFont=* Bold,
  	%BoldItalicFont=* Bold Italic,
  	SizeFeatures={Size=10.5},
]{FiraCode}

\usepackage[math-style=TeX]{unicode-math}
%\setmathfont{XITS Math}
\setmathfont[Path=./../fuentes/]{XITS Math}
\setmathfont[
	Path=./../fuentes/,
	range=up/{latin,Latin, num},
]{Equity Text A Regular}
\setmathfont[
	Path=./../fuentes/,
	range=it/{latin,Latin},
]{Equity Text A Italic}
\setoperatorfont\symup

\defaultfontfeatures{Ligatures=TeX,Numbers=Lining}


% ---------------------------------------------------------------------------------------
% 	CONFIGURACIÓN DEL ÍNDICE
% ---------------------------------------------------------------------------------------

\usepackage{titletoc}

\contentsmargin[0cm]{0cm}
\titlecontents{chapter}[0em]{\vskip12pt\bfseries\sffamily}
{\thecontentslabel\enspace}
{\hspace{1.05em}}
{ \hfill\contentspage}[\vskip 6pt]

\titlecontents{section}[1em]{\sffamily}
{\thecontentslabel\enspace}
{}
{\titlerule*[1pc]{.}\quad\contentspage}[\vskip 4pt]

\titlecontents{subsection}[2em]{\sffamily}
{\thecontentslabel\enspace}
{}
{\titlerule*[1pc]{.}\quad\contentspage}[\vskip 3pt]

\titlecontents{subsubsection}[4em]{\sffamily}
{\thecontentslabel\enspace}
{}
{\titlerule*[1pc]{.}\quad\contentspage}[\vskip 3pt]

\usepackage{etoolbox}
\pretocmd{\contentsname}{\sffamily}{}{}

% ---------------------------------------------------------------------------
% 	TÍTULOS DE PARTES Y SECCIONES
% ---------------------------------------------------------------------------

\usepackage{titlesec}

% Estilo de los títulos de las partes
\titleformat{\part}[hang]{\Huge\bfseries\sffamily}{\thepart\hspace{20pt}\textcolor{500}{|}\hspace{20pt}}{0pt}{\Huge\bfseries}
\titlespacing*{\part}{0cm}{-2em}{2em}[0pt]

% Reiniciamos el contador de secciones entre partes (opcional)
\makeatletter
\@addtoreset{section}{part}
\makeatother

% Estilo de los títulos de las secciones, subsecciones y subsubsecciones
\titleformat{\section}
  {\LARGE\bfseries\sffamily}{\thesection}{0.5em}{}

\titleformat{\subsection}
  {\Large\sffamily}{\thesubsection}{0.25em}{}

\titleformat{\subsubsection}
  {\large\sffamily}{\thesubsubsection}{0.25em}{}

 \usepackage[spanish]{babel}

\linespread{1.3} % Line spacing
\setlength{\parskip}{9pt}

\usepackage[bottom]{footmisc}

\renewcommand*\footnoterule{}


\setlength\parindent{0pt} % Uncomment to remove all indentation from paragraphs

\graphicspath{{Pictures/}} % Specifies the directory where pictures are stored

\usepackage[font=sf]{caption}

\begin{document}

\section*{Guía para la Demo de IPFS}
\pagenumbering{gobble}

Asumimos que tenemos IPFS instalado.

Inicializamos un nodo IPFS con \texttt{ipfs\ init}. Generará el
\emph{keypair} y creará un repositorio.

\texttt{peer\ identity} es el \emph{hash} de la clave pública.

Accedemos al directorio \textit{Demo} que contiene algunos archivos.

\texttt{ipfs\ add\ -r\ .} añade todo el directorio a IPFS. Nos muestra
los \emph{hashes} de los archivos.

Si dos archivos tienen el mismo \emph{hash} es porque son iguales.

\texttt{ipfs\ ls\ \textless{}hash\ del\ directorio\textgreater{}} nos
muestra los archivos del directorio que acabamos de añadir.

\texttt{ipfs\ cat\ \textless{}hash\ de\ un\ archivo\textgreater{}} nos
muestra el contenido del archivo. Podemos hacer
\texttt{ipfs\ cat\ \textless{}hash\ de\ un\ archivo\textgreater{}\ \textgreater{}\ \textless{}nombre\ archivo\ local\textgreater{}}
para guardar los contenidos en un archivo que podemos abrir.

Podemos iniciar el \emph{daemon} IPFS con \texttt{ipfs\ daemon}, que
iniciará el servidor IPFS y nos conectará con el resto de la red.

Si hacemos \texttt{ipfs\ id} nos mostrará información sobre nuestro nodo
IPFS, incluyendo la clave pública.

\texttt{ipfs\ swarm\ peers} nos muestra a qué otros nodos estamos
conectados. Podemos obtener información de estos nodos con
\texttt{ipfs\ id\ \textless{}hash\textgreater{}}.

Podemos acceder al servidor desde
\texttt{localhost:8080/ipfs/\textless{}hash\textgreater{}}, siendo
\texttt{\textless{}hash\textgreater{}} el hash del archivo al que
queremos acceder.

Podemos acceder igualmente al contenido de nuestro ordenador desde otro
\emph{gateway} de IPFS.

Por ejemplo, al mismo archivo de antes podemos acceder con su
\emph{hash}. Un \emph{gateway} es
\texttt{gateway.ipfs.io/ipfs/\textless{}hash\textgreater{}}. Lo que esto
hace es hará una petición, buscando en la DHT el archivo, contactará con
mi ordenador y bajará el archivo.

De igual forma, utilizando nuestro servidor local podemos acceder a
archivos que se encuentran en otros nodos. Por ejemplo, si queremos
acceder a una copia de la Wikipedia en IPFS simplemente hacemos
\texttt{localhost:8080/ipfs/\textless{}hash\ de\ la\ wikipedia\textgreater{}}.

Podemos acceder a la interfaz web con \texttt{127.0.0.1:5000/webui}.


\end{document}
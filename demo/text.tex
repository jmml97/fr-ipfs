Asumimos que tenemos IPFS instalado.

Inicializamos el nodo IPFS \texttt{ipfs\ init}. Generará el
\emph{keypair} y creará un repositorio.

\texttt{peer\ identity} es el \emph{hash} de la clave pública.

Directorio demo con algunos archivos.

\texttt{ipfs\ add\ -r\ .} añade todo el directorio a IPFS. Nos muestra
los \emph{hashes} de los archivos.

Si dos archivos tienen el mismo \emph{hash} es porque son iguales.

\texttt{ipfs\ ls\ \textless{}hash\ del\ directorio\textgreater{}} nos
muestra los archivos del directorio que acabamos de añadir.

\texttt{ipfs\ cat\ \textless{}hash\ de\ un\ archivo\textgreater{}} nos
muestra el contenido del archivo. Podemos hacer
\texttt{ipfs\ cat\ \textless{}hash\ de\ un\ archivo\textgreater{}\ \textgreater{}\ \textless{}nombre\ archivo\ local\textgreater{}}
para guardar los contenidos en un archivo que podemos abrir.

Podemos iniciar el \emph{daemon} IPFS con \texttt{ipfs\ daemon}, que
iniciará el servidor IPFS.

Si hacemos \texttt{ipfs\ id} nos mostrará información sobre nuestro nodo
IPFS, incluyendo la clave pública.

\texttt{ipfs\ swarm\ peers} nos muestra a qué otros nodos estamos
conectados. Podemos obtener información de estos nodos con
\texttt{ipfs\ id\ \textless{}hash\textgreater{}}.

Podemos acceder al servidor desde
\texttt{localhost:8080/ipfs/\textless{}hash\textgreater{}}, siendo
\texttt{\textless{}hash\textgreater{}} el hash del archivo al que
queremos acceder.

Podemos acceder igualmente al contenido de nuestro ordenador desde otro
\emph{gateway} de IPFS.

Por ejemplo, al mismo archivo de antes podemos acceder con su
\emph{hash}. Un \emph{gateway} es
\texttt{gateway.ipfs.io/ipfs/\textless{}hash\textgreater{}}. Lo que esto
hace es hará una petición, buscando en la DHT el archivo, contactará con
mi ordenador y bajará el archivo.

De igual forma, utilizando nuestro servidor local podemos acceder a
archivos que se encuentran en otros nodos. Por ejemplo, si queremos
acceder a una copia de la Wikipedia en IPFS simplemente hacemos
\texttt{localhost:8080/ipfs/\textless{}hash\ de\ la\ wikipedia\textgreater{}}.

Podemos acceder a la interfaz web con \texttt{127.0.0.1:5000/webui}.

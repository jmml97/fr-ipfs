%%%%%%%%%%%%%%%%%%%%%%%%%%%%%%%%%%%%%%%%%
% Simple Sectioned Essay Template
% LaTeX Template
%
% This template has been downloaded from:
% http://www.latextemplates.com
%
% Note:
% The \lipsum[#] commands throughout this template generate dummy text
% to fill the template out. These commands should all be removed when 
% writing essay content.
%
%%%%%%%%%%%%%%%%%%%%%%%%%%%%%%%%%%%%%%%%%

%----------------------------------------------------------------------------------------
%	PACKAGES AND OTHER DOCUMENT CONFIGURATIONS
%----------------------------------------------------------------------------------------

\documentclass[12pt]{article} % Default font size is 12pt, it can be changed here

\usepackage{geometry} % Required to change the page size to A4
\geometry{a4paper, left={40mm}, right={40mm}, top={25mm}, bottom={25mm},includehead,includefoot} % Set the page size to be A4 as opposed to the default US Letter

\usepackage{graphicx} % Required for including pictures

\usepackage{float} % Allows putting an [H] in \begin{figure} to specify the exact location of the figure
\usepackage{wrapfig} % Allows in-line images such as the example fish picture

\usepackage{lipsum} % Used for inserting dummy 'Lorem ipsum' text into the template

\usepackage{amsfonts}

\usepackage{listings}

% ---------------------------------------------------------------------------------------
%	TIPOGRAFÍA
% ---------------------------------------------------------------------------------------

\usepackage[no-math]{fontspec}
\setmainfont[
	Path=./../fuentes/, 
	UprightFont=* Regular, 
	ItalicFont=* Italic,
	BoldFont=* Bold,
  	BoldItalicFont=* Bold Italic,
]{Equity Text A}
\setsansfont[
	Path=./../fuentes/, 
	UprightFont=* Regular, 
	ItalicFont=* Italic,
	BoldFont=* Bold,
  	BoldItalicFont=* Bold Italic,
]{Concourse T3}
\setmonofont[
	Path=./../fuentes/, 
	UprightFont=* Regular, 
	ItalicFont=* Italic,
	BoldFont=* Bold,
  	BoldItalicFont=* Bold Italic,
  	SizeFeatures={Size=10.5},
]{Hack}

\usepackage[math-style=TeX]{unicode-math}
%\setmathfont{XITS Math}
\setmathfont[Path=./../fuentes/]{XITS Math}
\setmathfont[
	Path=./../fuentes/,
	range=up/{latin,Latin, num},
]{Equity Text A Regular}
\setmathfont[
	Path=./../fuentes/,
	range=it/{latin,Latin},
]{Equity Text A Italic}
\setoperatorfont\symup

\defaultfontfeatures{Ligatures=TeX,Numbers=Lining}


% ---------------------------------------------------------------------------------------
% 	CONFIGURACIÓN DEL ÍNDICE
% ---------------------------------------------------------------------------------------

\usepackage{titletoc}

\contentsmargin[0cm]{0cm}
\titlecontents{chapter}[0em]{\vskip12pt\bfseries\sffamily}
{\thecontentslabel\enspace}
{\hspace{1.05em}}
{ \hfill\contentspage}[\vskip 6pt]

\titlecontents{section}[1em]{\sffamily}
{\thecontentslabel\enspace}
{}
{\titlerule*[1pc]{.}\quad\contentspage}[\vskip 4pt]

\titlecontents{subsection}[2em]{\sffamily}
{\thecontentslabel\enspace}
{}
{\titlerule*[1pc]{.}\quad\contentspage}[\vskip 3pt]

\titlecontents{subsubsection}[4em]{\sffamily}
{\thecontentslabel\enspace}
{}
{\titlerule*[1pc]{.}\quad\contentspage}[\vskip 3pt]

\usepackage{etoolbox}
\pretocmd{\contentsname}{\sffamily}{}{}

% ---------------------------------------------------------------------------
% 	TÍTULOS DE PARTES Y SECCIONES
% ---------------------------------------------------------------------------

\usepackage{titlesec}

% Estilo de los títulos de las partes
\titleformat{\part}[hang]{\Huge\bfseries\sffamily}{\thepart\hspace{20pt}\textcolor{500}{|}\hspace{20pt}}{0pt}{\Huge\bfseries}
\titlespacing*{\part}{0cm}{-2em}{2em}[0pt]

% Reiniciamos el contador de secciones entre partes (opcional)
\makeatletter
\@addtoreset{section}{part}
\makeatother

% Estilo de los títulos de las secciones, subsecciones y subsubsecciones
\titleformat{\section}
  {\LARGE\bfseries\sffamily}{\thesection}{0.5em}{}

\titleformat{\subsection}
  {\Large\sffamily}{\thesubsection}{0.25em}{}

\titleformat{\subsubsection}
  {\large\sffamily}{\thesubsubsection}{0.25em}{}

 \usepackage[spanish]{babel}

\linespread{1.3} % Line spacing

%----------------------------------------------------------------------------------------
%	CABECERAS Y PIES DE PÁGINA
%----------------------------------------------------------------------------------------

\usepackage{fancyhdr}
 
\pagestyle{fancy}
\fancyhf{}
\rfoot{\sffamily Página \thepage\ de \pageref{LastPage}}
\lfoot{\sffamily La web distribuida: el protocolo IPFS}

\renewcommand{\headrulewidth}{0pt}
\renewcommand{\footrulewidth}{0.5pt}

\usepackage{lastpage}

\setlength\parindent{0pt} % Uncomment to remove all indentation from paragraphs

\graphicspath{{Pictures/}} % Specifies the directory where pictures are stored

\begin{document}

%----------------------------------------------------------------------------------------
%	PÁGINA DEL TÍTULO
%----------------------------------------------------------------------------------------

\begin{titlepage}
\sffamily

\newcommand{\HRule}{\rule{\linewidth}{0.5mm}} % Defines a new command for the horizontal lines, change thickness here

\center % Center everything on the page

\textsc{\LARGE Universidad de Granada}\\[1.5cm] % Name of your university/college
\textsc{\Large Doble Grado en Ingeniería Informática y Matemáticas}\\[0.5cm] % Major heading such as course name
\textsc{\large Fundamentos de Redes}\\[0.5cm] % Minor heading such as course title

\HRule \\[1cm]
{ \huge \bfseries La web distribuida: el protocolo IPFS}\\[0.4cm] % Title of your document
\HRule \\[1.5cm]

\begin{minipage}{0.4\textwidth}
\begin{flushleft} \large
\emph{Autores:}\\
José María Martín Luque\\
Adolfo Soto Werner % Your name
\end{flushleft}
\end{minipage}
~
\begin{minipage}{0.4\textwidth}
\begin{flushright} \large
\emph{Profesor:} \\
Antonio Ruiz Moya\\ % Supervisor's Name
\hfill\\
\end{flushright}
\end{minipage}\\[4cm]

{\large \today}\\[3cm] % Date, change the \today to a set date if you want to be precise

%\includegraphics{Logo}\\[1cm] % Include a department/university logo - this will require the graphicx package

\vfill % Fill the rest of the page with whitespace

\end{titlepage}

%----------------------------------------------------------------------------------------
%	ÍNDICE
%----------------------------------------------------------------------------------------

\tableofcontents % Include a table of contents

\newpage % Begins the essay on a new page instead of on the same page as the table of contents 

%----------------------------------------------------------------------------------------
%	INTRODUCTION
%----------------------------------------------------------------------------------------

\section{Introducción}


%----------------------------------------------------------------------------------------
%	CONTENIDO
%----------------------------------------------------------------------------------------

\section{Problemas de HTTP} % (fold)
\label{sec:problemas_de_http}

\lipsum[10]

% section problemas_de_http (end)

\section{La web distribuida} % (fold)
\label{sec:la_web_distribuida}

\subsection{Ventajas de la web distribuida} % (fold)
\label{sub:ventajas_de_la_web_distribuida}

\lipsum[6]

\subsubsection{Hola caracola}

\lipsum[3]

% subsection ventajas_de_la_web_distribuida (end)
% section la_web_distribuida (end)

\section{El protocolo IPFS} % (fold)
\label{sec:el_protocolo_ipfs}

% section el_protocolo_ipfs (end)

\section{La web distribuida en la actualidad} % (fold)
\label{sec:la_web_distribuida_en_la_actualidad}

% section la_web_distribuida_en_la_actualidad (end)

%----------------------------------------------------------------------------------------
%	BIBLIOGRAFÍA
%----------------------------------------------------------------------------------------

\newpage
\begin{thebibliography}{99} % Bibliography - this is intentionally simple in this template

\bibitem[Figueredo and Wolf, 2009]{Figueredo:2009dg}
Figueredo, A.~J. and Wolf, P. S.~A. (2009).
\newblock Assortative pairing and life history strategy - a cross-cultural
  study.
\newblock {\em Human Nature}, 20:317--330.
 
\end{thebibliography}

%----------------------------------------------------------------------------------------

\end{document}